\section{File Manipulation}

\subsection{Exploring}
\begin{center}
\begin{longtable}{l|l}
 \verb!:Exp(lore)! & file explorer (note: capital E)\\
 \verb!:Sex(plore)! & file explorer in split window\\
 \verb!:ls! & list of buffers\\
 \verb!:cd ..! & move to parent directory\\
 \verb!:args! & list of files\\
 \verb!:lcd %:p:h! & change to directory of current file\\
 \verb!:autocmd BufEnter * lcd %:p:h! & change to directory of current file automatically \footnote{Script required: bufexplorer.vim \url{http://www.vim.org/script.php?script\_id=42}}\\
 \verb!\be! & buffer explorer list of buffers\\
 \verb!\bs! & buffer explorer (split window)
\end{longtable}
\end{center}

\subsection{Opening files \& other tricks}
\begin{center}
\begin{longtable}{l|l}
 \verb!gf! & open file name under cursor (SUPER)\\
 \verb!:nnoremap gF :view ! & open file under cursor, create if necessary\\
 \verb!ga! & display hex,ascii value of char under cursor\\
 \verb!ggVGg?! & rot13 whole file\\
 \verb!ggg?G! & rot13 whole file (quicker for large file)\\
 \verb!:8 | normal VGg?! & rot13 from line 8\\
 \verb!:normal 10GVGg?! & rot13 from line 8\\
 \verb!<c-a>,<c-x>! & increment, decrement number under cursor\\
 \verb!<c-r>=5*5! & insert 25 into text (mini-calculator)\\
 \verb!:e main_<tab>! & tab completes\\
 \verb!main_<c-x><c-f>! & include NAME of file in text (insert mode)
\end{longtable}
\end{center}

\subsection{Multiple files management}
\begin{center}
\begin{longtable}{l|l}
 \verb!:bn! & goto next buffer\\
 \verb!:bp! & goto previous buffer\\
 \verb!:wn! & save file and move to next (super)\\
 \verb!:wp! & save file and move to previous\\
 \verb!:bd! & remove file from buffer list (super)\\
 \verb!:bun! & buffer unload (remove window but not from list)\\
 \verb!:badd file.c! & file from buffer list\\
 \verb!:b 3! & go to buffer 3\\
 \verb!:b main! & go to buffer with main in name eg main.c (ultra)\\
 \verb!:sav php.html! & save current file as php.html and "move" to php.html\\
 \verb?:sav! %<.bak? & save current file to alternative extension (old way)\\
 \verb?:sav! %:r.cfm? & save current file to alternative extension\\
 \verb!:sav %:s/fred/joe/! & do a substitute on file name\\
 \verb!:sav %:s/fred/joe/:r.bak2! & do a substitute on file name \& ext.\\
 \verb?:!mv % %:r.bak? & rename current file (DOS use rename or del)\\
 \verb?:e!? & return to unmodified file\\
 \verb!:w c:/aaa/%! & save file elsewhere\\
 \verb!:e #! & edit alternative file (also \verb!ctrl-^!)\\
 \verb!:rew! & return to beginning of edited files list (:args)\\
 \verb!:brew! & buffer rewind\\
 \verb!:sp fred.txt! & open fred.txt into a split\\
 \verb!:sball,:sb! & split all buffers (super)\\
 \verb!:scrollbind! & in each split window\\
 \verb!:map <F5> :ls<CR>:e #! & pressing F5 lists all buffers, just type number\\
 \verb!:set hidden! & allows to change buffer w/o saving current buffer
 \end{longtable}
\end{center}

\subsection{File-name manipulation}
\begin{center}
\begin{longtable}{l|l}
 \verb!:h filename-modifiers! & help\\
 \verb!:w %! & write to current file name\\
 \verb!:w %:r.cfm! & change file extention to .cfm\\
 \verb?:!echo %:p? & full path \& file name\\
 \verb?:!echo %:p:h? & full path only\\
 \verb!<C-R>%! & insert filename (insert mode)\\
 \verb!"%p! & insert filename (normal mode)\\
 \verb!/<C-R>%! & search for file name in text
\end{longtable}
\end{center}

\subsection{Command over multiple files}
\begin{center}
\begin{longtable}{l|l}
 \verb!:argdo %s/foo/bar/e! & operate on all files in :args\\
 \verb!:bufdo %s/foo/bar/e! & operate on all buffers\\
 \verb!:windo %s/foo/bar/e! & operate on all windows\\
 \verb?:argdo exe '%!sort'|w!? & include an external command
\end{longtable}
\end{center}

\subsection{Sessions (set of files)}
\begin{center}
\begin{longtable}{l|l}
 \verb!gvim file1.c file2.c lib/lib.h lib/lib2.h! & load files for "session"\\
 \verb!:mksession! & create a session file (default session.vim)\\
 \verb!gvim -S Session.vim! & reload all files
\end{longtable}
\end{center}

\subsection{Modelines}
\begin{center}
\begin{longtable}{l|l}
 \verb!vim:noai:ts=2:sw=4:readonly:! & makes readonly\\
 \verb!vim:ft=html:! & says use HTML syntax highlighting\\
 \verb!:h modeline! & help with modelines
\end{longtable}
\end{center}

\subsubsection{Creating your own GUI Toolbar entry}

\begin{verbatim}
amenu  Modeline.Insert\ a\ VIM\ modeline
       \ <esc><esc>ggOvim:ff=unix ts=4 ss=4<CR>vim60:fdm=marker<esc>gg
\end{verbatim}

\subsection{Markers \& moving about}
\begin{center}
\begin{longtable}{l|l}
 \verb!'.! & jump to last modification line (SUPER)\\
 \verb!`.! & jump to exact spot in last modification line\\
 \verb!g;! & cycle through recent changes (oldest first) \footnote{(new in vim 6.3)}\\
 \verb!g,! & reverse direction \footnote{(new in vim 6.3)}\\
 \verb!:changes! & show entire list of changes\\
 \verb!:h changelist! & help for above\\
 \verb!<C-O>! & retrace your movements in file (starting from most recent)\\
 \verb!<C-I>! & retrace your movements in file (reverse direction)\\
 \verb!:ju(mps)! & list of your movements\\
 \verb!:help jump-motions! & explains jump motions\\
 \verb!:history! & list of all your commands\\
 \verb!:his c! & commandline history\\
 \verb!:his s! & search history\\
 \verb!q/! & search history window (puts you in full edit mode)\\
 \verb!q:! & commandline history window (puts you in full edit mode)\\
 \verb!:! & history Window
 \end{longtable}
\end{center}

\subsection{Editing/moving within insert mode}
\begin{center}
\begin{longtable}{l|l}
\verb!<C-U>! & delete all entered\\
\verb!<C-W>! & delete last word\\
\verb!<HOME><END>! & beginning/end of line\\
\verb!<C-LEFTARROW><C-RIGHTARROW>! & jump one word backwards/forwards\\
\verb!<C-X><C-E>,<C-X><C-Y>! & scroll while staying put in insert
\end{longtable}
\end{center}

\subsection{Abbreviations \& maps}
\begin{verbatim}
:map <f7>   :'a,'bw! c:/aaa/x
:map <f8>   :r c:/aaa/x
:map <f11>  :.w! c:/aaa/xr<CR>
:map <f12>  :r c:/aaa/xr<CR>
\end{verbatim}
\begin{center}
\begin{longtable}{l|l}
\verb!:ab php! & list of abbreviations beginning php\\
\verb!:map ,! & list of maps beginning ,\\
\verb!set wak=no! & allow use of F10 for win32 mapping (:h winaltkeys)\\
\verb!<CR>! & enter\\
\verb!<ESC>! & escape\\
\verb!<BACKSPACE>! & backspace\\
\verb!<LEADER>! & backslash\\
\verb!<BAR>! & |\\
\verb!<SILENT>! & execute quietly\\
\verb!iab phpdb exit("<hr>Debug <C-R>a  ");! & yank all variables into register a
\end{longtable}
\end{center}

\subsubsection{Display RGB colour under the cursor eg \#445588}

\begin{verbatim}
:nmap <leader>c :hi Normal guibg=#<c-r>=expand("<cword>")<cr><cr>
\end{verbatim}
